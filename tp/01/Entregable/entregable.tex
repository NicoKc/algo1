\documentclass{article}
\input{../Latex/Algo1Macros.tex}

\begin{document}

\begin{proc}{esSeñal}{\In s: \TLista{\ent}, \In prof: \ent, \In freq: \ent, out result: \bool}{}
\pre{\longitud{s} \geq 0 \wedge prof > 0 \wedge freq > 0}
\post{ \\
    result = frecuenciaEnRango(freq) \yLuego \\
        profundidadCorrecta(s) \yLuego \\
        ningunaMuestraSuperaLaProfundidad(s, prof) \yLuego \\
        duraMasDeUnSegundo(s, freq)
}
\end{proc}

\pred{profundidadCorrecta}{prof: \ent}{ freq \in [8,16,32] }

\pred{frecuenciaEnRango}{freq: \ent}{ freq \in [8,32] }

\pred{ningunaMuestraSuperaLaProfundidad}{s: \TLista{\ent}, p: \ent}{

    \paraTodo{0}{i}{|s|}

    \implicaLuego (-2)^{p-1} \leq s[i] \leq 2^{p-1}-1

}

\pred{duraMasDeUnSegundo}{s: \TLista{\ent}, freq: \ent}{
    \frac{\longitud{s}}{(freq\cdot 1000)}\geq 1
}

\pred{esSeñalAux}{s: \TLista{\ent}, prof: \ent, freq: \ent}{

    \longitud{s} \geq 0 \yLuego

    frecuenciaEnRango(freq) \yLuego

    profundidadCorrecta(s) \yLuego

    ningunaMuestraSuperaLaProfundidad(s, prof) \yLuego

    duraMasDeUnSegundo(s, freq)

}

\begin{proc}{seEnojó?}{\In s: señal, \In umbral: \ent, \In prof: \ent, \In freq: \ent, out result: \bool}{}
    \pre{ umbral > 0 \yLuego esSeñalAux(s, prof, freq) }
    \post{ \\
        result = umbralEnRango(umbral, prof) \yLuego \\
            existeUnaSubsecuenciaQueSuperaUmbral(s, freq, umbral)
    }
    \end{proc}

    \pred{umbralEnRango}{umbral: \ent, p: \ent}{ umbral \geq 2^{p-1}-1 }

    \pred{existeUnaSubsecuenciaQueSuperaUmbral}{s: señal, freq: \ent, u: \ent}{

        \existe{d,h}{0}{\longitud{s}}
        \wedge
        (d < h) \wedge
        ((d + freq * 1000 * 5) < h) \yLuego (

            \paraTodo{i}{0}{\longitud{subseq(s,d,h)}} 
            \implicaLuego
            abs(subseq(s,d,h)[i]) > umbral
        )
    }

    \aux{abs}{x: \ent}{\ent}{\IfThenElse{x > 0}{x}{-x}}

    \begin{proc}{esReuniónVálida?}{\In r: reunion,
        \In prof: \ent,
        \In freq: \ent,
        out result: \bool}{}
    \pre{ \longitud{r} > 0 \wedge prof > 0 \wedge freq > 0 }
    \post{ \\
        result = contieneSeñalesValidas(r, prof, freq) \yLuego \\
            lasLongitudesDeSeñalSonIguales(r) \yLuego \\
            todosHablantesDistintos(r) \yLuego \\
            losHablantesEstanEnRangosDe0ANMenos1(r)
    }
    \end{proc}
    
    \pred{contieneSeñalesValidas}{r: reunion,
        prof: \ent,
        freq: \ent}
    {
        \paraTodo{i}{0}{\longitud{r}}
        \implicaLuego esSeñalAux(r[i]_{0}, prof, freq)
    }
    
    \pred{lasLongitudesDeSeñalSonIguales}{r: reunion}
    {
        \paraTodo{i,j}{0}{\longitud{r} \wedge i \neq j}
        \implicaLuego (\longitud{r[i]_{0}} = \longitud{r[j]_{0}})
    }
    
    \pred {todosHablantesDistintos}{r: reunion}{
        \paraTodo{i,j}{0}{\longitud{r} \wedge i \neq j}
        \implicaLuego (r[i]_{1} \neq r[j]_{1})
    }
    
    \pred {losHablantesEstanEnRangosDe0ANMenos1}{r: reunion}{
        \paraTodo{i}{0}{\longitud{r}}
        \implicaLuego 0 \leq r[i]_{1} < \longitud{r}
    }
    
    \pred {esReuniónVálidaAux}{r: reunion,
        prof: \ent,
        freq: \ent}{
    
        contieneSeñalesValidas(r, prof, freq) \yLuego
    
        lasLongitudesDeSeñalSonIguales(r) \yLuego
    
        todosHablantesDistintos(r) \yLuego
    
        losHablantesEstanEnRangosDe0ANMenos1(r)
    }

    \begin{proc}{acelerar}{\Inout r: reunion,
        \In prof: \ent,
        \In freq: \ent}{}
    \pre{ esReuniónVálidaAux(r, prof, freq) \wedge 
        r_{0} = r }
    \post{ \\
        esReuniónVálidaAux(r, prof, freq) \yLuego \\
            \longitud{r} = \longitud{r_{0}} \yLuego \\
            lasSeñalesTieneLaMitadDeMuestras(r, r_{0}) \yLuego \\
            losImpares(r, r_{0})
    }
    \end{proc}
    
    \pred{lasSeñalesTieneLaMitadDeMuestras}{r: reunion,
        r_{0}: reunion}
    {
    
        \paraTodo{i}{0}{\longitud{r}}
        \implicaLuego
        \IfThenElse{esPar(\longitud{r[0]_{0}})}
        {\longitud{r[i]_{0}} = \frac{\longitud{r[0]_{0}}}{2}}
        {\longitud{r[i]_{0}} = \frac{\longitud{r[0]_{0}}-1}{2}}
    
    }
    
    \pred{losImpares}{r: reunion,
        r_{0}: reunion}
    {
        \paraTodo{i}{0}{\longitud{r}}
        \implicaLuego (
    
            \existe{j}{0}{\longitud{r}} \yLuego (
                r[i]_{1} = r[j]_{1}
            ) \yLuego (
    
                \paraTodo{q}{0}{\longitud{r_{0}[i]_{0}} \wedge
                (\neg esPar(q))}
                \implicaLuego (
                    r_{0}[i]_{0}[q] = r[j]_{0}[\frac{q-1}{2}]
                )
            )
        )
    }

    \begin{proc}{ralentizar}{\Inout r: reunion,
        \In prof: \ent,
        \In freq: \ent}{}
    \pre{ esReuniónVálidaAux(r, prof, freq) \wedge 
        r_{0} = r }
    \post{ \\
        esReuniónVálidaAux(r, prof, freq) \yLuego \\
            \longitud{r} = \longitud{r_{0}} \yLuego \\
            lasSeñalesTienenElDobleDeMuestras(r, r_{0}) \yLuego \\
            promedioEntrePares(r, r_{0})
    }
    \end{proc}
    
    \pred{lasSeñalesTienenElDobleDeMuestras}{r: reunion,
        r_{v}: reunion}
    {
    
        \paraTodo{i}{0}{\longitud{r_{v}}}
        \implicaLuego
            (2 \cdot \longitud{r_{v}[i]_{0}}) = (\longitud{r[i]_{0}} + 1)
    }
    
    \pred{promedioEntrePares}{r: reunion,
        r_{v}: reunion}
    {
    
        \paraTodo{i}{0}{\longitud{r}}
        \implicaLuego (
    
            \existe{j}{0}{\longitud{r_{0}}} \yLuego (
                r[i]_{1} = r_{v}[j]_{1}
            ) \yLuego (
                
                \paraTodo{q}{0}{\longitud{r[i]_{0}}}
                \implicaLuego
    
                \IfThenElse{esPar(q)}{
                    r[j]_{0}[q] = r_{v}[i]_{0}[\frac{q}{2}]
                }{
                    r[j]_{0}[q] = \frac{r_{v}[i]_{0}[\frac{q-1}{2}] + r_{v}[i]_{0}[\frac{q+1}{2}]}{2}
                }
            )
        )
    }

    \begin{proc}{tonosDeVozElevados}{\Inout r: reunion,
        \In freq: \ent,
        \In prof: \ent,
        \Out hablantes: \TLista{hablante}}{}
    \pre{ esReuniónVálidaAux(r, prof, freq) }
    \post{ \\
        siPertenecenAHablantesElPromedioDeAmplitudEsMasGrandeOIgualQueElResto(r, hablantes) \yLuego \\
            losHablantesPertenecenALaReunión(r, hablantes) \yLuego \\
            losHablantesNoSeRepiten(hablantes)
    }
    \end{proc}
    
    \pred{siPertenecenAHablantesElPromedioDeAmplitudEsMasGrandeOIgualQueElResto}{r: reunion,
        hs: \TLista{hablante}}
    {
    
        \paraTodo{i}{0}{\longitud{hs}}
        \implicaLuego
    
            (r[i]_{1} \in hs 
            \wedge
            elPromedioDeAmplitudEsMasGrandeOIgualQueElResto(r, r[i]_{0}))
    
            \vee
    
            (r[i]_{1} \notin hs 
            \wedge
            \neg elPromedioDeAmplitudEsMasGrandeOIgualQueElResto(r, r[i]_{0}))
    
    }
    
    \pred{losHablantesPertenecenALaReunión}{r: reunion,
        hs: \TLista{hablante}}
    {
    
        \paraTodo{i}{0}{\longitud{hs}}
        \implicaLuego (
            \existe{j}{0}{\longitud{r}} \yLuego (
                hs[i] = r[j]_{1}
            )
        )
    }
    
    \pred{losHablantesNoSeRepiten}{r: reunion,
        hs: \TLista{hablante}}
    {
    
        \paraTodo{i}{0}{\longitud{hs}}
        \implicaLuego (
            \#apariciones(hs, hs[i]) = 1
        )
    }
    
    \pred{elPromedioDeAmplitudEsMasGrandeOIgualQueElResto}{r: reunion,
        s: señal}
    {
    
        \paraTodo{i}{0}{\longitud{r}}
        \implicaLuego (
            tonoDeVoz(s) \geq tonoDeVoz(r[i]_{0})
        )
    }
    
    \aux{tonoDeVoz}{s: señal}{\ent}{
        sumaDelValorAbsolutoDeAmplitudes(s)  div  \longitud{s}
    }
    
    \aux{sumaDelValorAbsolutoDeAmplitudes}{s: señal}{\ent}{
        \sum_{i=0}^{\longitud{s}} abs(s[i])
    }

    \begin{proc}{ordenar}{\Inout r: reunion,
        \In freq: \ent,
        \In prof: \ent}{}
    \pre{ esReuniónVálidaAux(r, prof, freq) \wedge r_{0} = r }
    \post{ \\
        esReuniónVálidaAux(r, prof, freq) \yLuego \\
            ordenadaDeMayorAMenorPorTonoDeVoz(r) \yLuego \\
            esUnaPermutación(r_{0}, r)
    }
    \end{proc}
    
    \pred{ordenadaDeMayorAMenorPorTonoDeVoz}{r: reunion}
    {
    
        \paraTodo{i}{1}{\longitud{r}}
        \implicaLuego
        tonoDeVoz(r[i-1]_{0}) \geq tonoDeVoz(r[i]_{0})
    
    }
    
    \pred{esUnaPermutación}{x: reunion,
        y: reunion}
    {
    
        \longitud{x} = \longitud{y} \yLuego
    
        \paraTodo{i}{0}{\longitud{x}}
        \implicaLuego (
    
            \existe{j}{0}{\longitud{y}} \yLuego (
                x[i]_{1} = y[j]_{1}
            ) \yLuego (
                x[i]_{0} = y[j]_{0}
            )
        )
    }
    
    \aux{tonoDeVoz}{s: señal}{\ent}{
        sumaDelValorAbsolutoDeAmplitudes(s)  div  \longitud{s}
    }
    
    \aux{sumaDelValorAbsolutoDeAmplitudes}{s: señal}{\ent}{
        \sum_{i=0}^{\longitud{s}} abs(s[i])
    }

    \begin{proc}{silencios}{\In s: señal,
        \In freq: \ent,
        \In prof: \ent,
        \Out intervalos: \TLista{intervalo}}{}
    \pre{ esSeñalAux(s, prof, freq) \wedge (umbral > 0) }
    \post{
    
        noHayIntervalosRepetidos(intervalos) \yLuego \\
        \paraTodo{i}{0}{\longitud{intervalos}} \implicaLuego (\\
        esSilencio(s, umbral, freq, intervalos[i])
        % finEsMayorQueInicio(intervalos[i]_{0}, intervalos[i]_{1}) \yLuego \\
        % estaDentroDeLaSeñal(s, intervalos[i]_{0}, intervalos[i]_{1}) \yLuego \\
        % esAlMenosUnDecimoDeSegundo(freq, intervalos[i]_{0}, intervalos[i]_{1}) \yLuego \\
        % entreIndicesNoPasaCiertoUmbral(s, umbral, intervalos[i]_{0}, intervalos[i]_{1}) \yLuego \\
        % losAdyacentesSuperanElUmbral(s, umbral, intervalos[i]_{0}, intervalos[i]_{1})
        )
    }
    \end{proc}
    
    \pred{noHayIntervalosRepetidos}{ins: \TLista{intervalo}}
    {
        \paraTodo{i}{0}{\longitud{ins}}
        \implicaLuego (
            \#apariciones(ins, e) = 1
        )
    }
    
    \pred{finEsMayorQueInicio}{inicio: \ent,
        fin: \ent}
    {
        fin > inicio
    }
    
    \pred{estaDentroDeLaSeñal}{s: señal,
        inicio: \ent,
        fin: \ent,}
    {
        (inicio \geq 0) \wedge (fin < |s|)
    }
    
    \pred{esAlMenosUnDecimoDeSegundo}{freq: \ent,
        inicio: \ent,
        fin: \ent,}
    {
        (fin - inicio + 1) \geq (frecuencia * 100)
    }
    
    \pred{losAdyacentesSuperanElUmbral}{s: señal,
        umbral: \ent,
        inicio: \ent,
        fin: \ent}
    {
    
        (
            (inicio = 0) \oLuego (
                (
                    inicio-1 \geq 0 
                ) \yLuego (
                    s[inicio-1] \geq umbral
                )
            )
        ) \wedge (
    
            (fin = \longitud{s} - 1) \oLuego (
                (
                    fin+1 < \longitud{s}
                ) \yLuego (
                    s[fin+1] \geq umbral
                )
            )
        )
    }
    
    \pred{entreIndicesNoPasaCiertoUmbral}{s: señal,
        umbral: \ent,
        inicio: \ent,
        fin: \ent}
    {
    
        \paraTodo{i}{inicio}{fin+1}
        \implicaLuego
        (abs(s[i]) \leq umbral)
    }
    
    \pred{esSilencio}{s: senal,
        umbral: \ent,
        freq: \ent,
        in: intervalo}
    {
    
        finEsMayorQueInicio(in_{0}, in_{1}) \yLuego
    
        estaDentroDeLaSeñal(s, in_{0}, in_{1}) \yLuego
    
        esAlMenosUnDecimoDeSegundo(freq, in_{0}, in_{1}) \yLuego
    
        entreIndicesNoPasaCiertoUmbral(s, umbral, in_{0}, in_{1}) \yLuego
    
        losAdyacentesSuperanElUmbral(s, umbral, in_{0}, in_{1})
    }
    

    \begin{proc}{hablantesSuperpuestos}{\In r: reunion,
        \In prof: \ent,
        \In freq: \ent,
        \In umbral: \ent,
        \Out result: \bool}{}
    \pre{ esReuniónVálidaAux(r, prof, freq) }
    \post{
        result = \neg noHayHablantesSuperpuestos(r, freq, umbral)
    }
    \end{proc}
    
    \pred{haySilencio}{s: señal,
        umbral: \ent,
        freq: \ent}
    {
        \existe{i,j}{0}{\longitud{s} \wedge (i < j)}
        \yLuego
        esSilencio(s, umbral, (i,j))
    }
    
    \pred{noHayHablantesSuperpuestos}{r: reunion,
        freq: \ent,
        umbral: \ent}
    {
    
        \paraTodo{i,j}{0}{\longitud{r} \wedge (i \neq j)}
        \implicaLuego
    
            \paraTodo{k,l}{0}{\longitud{r[i]_{0}}} \wedge k < l \implicaLuego
    
            \neg haySilencio(subseq(r[i]_{0}, k, l),  umbral, freq) \implicaLuego
            esSilencio(r[j]_{0}, umbral, freq, (k, l))
    }

    \begin{proc}{reconstruir}{\In s: $señal$,
        \In prof: \ent,
        \In freq: \ent,
        \Out señal: \bool}{}
    \pre{ esSeñalAux(s, prof, freq) }
    \post{
        esSeñalAux(result) \yLuego
    
            \longitud{s} = \longitud{result} \yLuego
    
            $enDondeNoSeaCeroDebenCoincidir(s, result)$ \yLuego
    
            $enDondeEsCeroDebeSerElPromedioDeSusVecinosNoNulos(s, result)$
    }
    \end{proc}
    
    \pred{enDondeNoSeaCeroDebenCoincidir}{original: señal,
        reconstruida: señal}
    {
    
        \paraTodo{i}{0}{\longitud{original}}
        \implicaLuego
    
            ($original[i]$ \neq 0) \yLuego 
    
            $(original[i] = reconstruida[i])$
    }
    
    \pred{enDondeEsCeroDebeSerElPromedioDeSusVecinosNoNulos}{original: señal,
        reconstruida: señal}
    {
    
        \paraTodo{i}{0}{\longitud{original}}
        \implicaLuego
    
            $(original[i] = 0)$ \yLuego 
    
            $reconstruida[i] = promedioDeVecinosNoNulos(original[i], reconstruida[i])$
    }
    
    \aux{promedioDeVecinosNoNulos}{s: $señal$,
        i: \ent}{\ent}
    {   
        \frac{(s[elIndiceNoNuloMasCercano(s, i)] + s[el2doIndiceNoNuloMasCercano(s, i)])}{2}
    }
    
    \aux{elIndiceNoNuloMasCercano}{s: $señal$,
        i: \ent}{\ent}
    {
    
        \IfThenElse{
            dist(i, indiceSiguienteNoNulo(s, i)) < dist(i, indiceAnteriorNoNulo(s, i))
        }{
    
            $indiceSiguienteNoNulo(s, i)$
        }{
    
            \IfThenElse{
                dist(i, indiceSiguienteNoNulo(s, i))> dist(i, indiceAnteriorNoNulo(s,i))
            }{
    
                $indiceAnteriorNoNulo(s, i)$
            }{
    
                $indiceAnteriorNoNulo(s, i)$ \vee $indiceSiguienteNoNulo(s, i)$
            }
        }
    }
    
    \aux{dist}{x: \ent,
        y: \ent}{\ent}{
            abs(x-y)
        }
    
    \aux{el2doIndiceNoNuloMasCercano}{s: $señal$,
        i: \ent}{\ent}{
    
            $elIndiceNoNuloMasCercano(setAt(s, elIndiceNoNuloMasCercano(s, i), 0))$
    }
    
    \aux{indiceAnteriorNoNulo}{s: $señal$,
        i: \ent}{\ent}{
        \sum_{p=0}^{i-1}
        \IfThenElse{esElPrimerAnteriorNoNulo(s, i, p)}{p}{0}
    }
    
    \pred{esElPrimerAnteriorNoNulo}{s: $señal$,
        i: \ent,
        p: \ent}
    {
        \paraTodo{j}{p}{i}
        \implicaLuego
        (s[j] = 0) \yLuego (s[p] \neq 0)
    }
    
    \aux{indiceSiguienteNoNulo}{s: $señal$,
        i: \ent}{\ent}
    {
        \sum_{p=i+1}^{\longitud{s}-1}
        \IfThenElse{esElPrimerSiguienteNoNulo(s, i, p)}{p}{0}
    }
    
    \pred{esElPrimerSiguienteNoNulo}{s: $señal$,
        i: \ent,
        p: \ent}
    {
        \paraTodo{j}{i}{p}
        \implicaLuego
        (s[j] = 0) \yLuego (s[p] \neq 0)
    }
    

\end{document}