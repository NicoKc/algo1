\documentclass{standalone}
\input{../Latex/Algo1Macros.tex}

\begin{document}

\begin{proc}{tonosDeVozElevados}{\Inout r: reunion,
    \In freq: \ent,
    \In prof: \ent,
    \Out hablantes: \TLista{hablante}}{}
\pre{ esReuniónVálidaAux(r, prof, freq) }
\post{ \\
    siPertenecenAHablantesElPromedioDeAmplitudEsMasGrandeOIgualQueElResto(r, hablantes) \yLuego \\
        losHablantesPertenecenALaReunión(r, hablantes) \yLuego \\
        losHablantesNoSeRepiten(hablantes)
}
\end{proc}

\pred{siPertenecenAHablantesElPromedioDeAmplitudEsMasGrandeOIgualQueElResto}{r: reunion,
    hs: \TLista{hablante}}
{

    \paraTodo{i}{0}{\longitud{hs}}
    \implicaLuego

        (r[i]_{1} \in hs 
        \wedge
        elPromedioDeAmplitudEsMasGrandeOIgualQueElResto(r, r[i]_{0}))

        \vee

        (r[i]_{1} \notin hs 
        \wedge
        \neg elPromedioDeAmplitudEsMasGrandeOIgualQueElResto(r, r[i]_{0}))

}

\pred{losHablantesPertenecenALaReunión}{r: reunion,
    hs: \TLista{hablante}}
{

    \paraTodo{i}{0}{\longitud{hs}}
    \implicaLuego (
        \existe{j}{0}{\longitud{r}} \yLuego (
            hs[i] = r[j]_{1}
        )
    )
}

\pred{losHablantesNoSeRepiten}{r: reunion,
    hs: \TLista{hablante}}
{

    \paraTodo{i}{0}{\longitud{hs}}
    \implicaLuego (
        \#apariciones(hs, hs[i]) = 1
    )
}

\pred{elPromedioDeAmplitudEsMasGrandeOIgualQueElResto}{r: reunion,
    s: señal}
{

    \paraTodo{i}{0}{\longitud{r}}
    \implicaLuego (
        tonoDeVoz(s) \geq tonoDeVoz(r[i]_{0})
    )
}

\aux{tonoDeVoz}{s: señal}{\ent}{
    sumaDelValorAbsolutoDeAmplitudes(s)  div  \longitud{s}
}

\aux{sumaDelValorAbsolutoDeAmplitudes}{s: señal}{\ent}{
    \sum_{i=0}^{\longitud{s}} abs(s[i])
}

\end{document}