% \documentclass{standalone}
% \input{../Latex/Algo1Macros.tex}

% \begin{document}

\begin{proc}{reconstruir}{\In s: $señal$,
    \In prof: \ent,
    \In freq: \ent,
    \Out señal: \bool}{}
\pre{ esSeñalAux(s, prof, freq) }
\post{
    esSeñalAux(result) \wedge

        (\longitud{s} = \longitud{result} \yLuego

        $enDondeNoSeaCeroDebenCoincidir$(s, result) \yLuego

        $enDondeEsCeroDebeSerElPromedioDeSusVecinosNoNulos$(s, result))
}
\end{proc}

\pred{enDondeEsCeroDebeSerElPromedioDeSusVecinosNoNulos}{original: señal,
    reconstruida: señal}
{

    \paraTodo{i}{0}{\longitud{original}}
    \implicaLuego

        $(original[i] = 0)$ \yLuego 

        $reconstruida[i] = promedioDeVecinosNoNulos(original[i], reconstruida[i])$
}

\aux{promedioDeVecinosNoNulos}{s: $señal$,
    i: \ent}{\ent}
{   
    \frac{(s[elIndiceNoNuloMasCercano(s, i)] + s[el2doIndiceNoNuloMasCercano(s, i)])}{2}
}

\aux{elIndiceNoNuloMasCercano}{s: $señal$,
    i: \ent}{\ent}
{

    \IfThenElse{
        dist(i, indiceSiguienteNoNulo(s, i)) < dist(i, indiceAnteriorNoNulo(s, i))
    }{

        $indiceSiguienteNoNulo(s, i)$
    }{

        \IfThenElse{
            dist(i, indiceSiguienteNoNulo(s, i))> dist(i, indiceAnteriorNoNulo(s,i))
        }{

            $indiceAnteriorNoNulo(s, i)$
        }{

            $indiceAnteriorNoNulo(s, i)$ \vee $indiceSiguienteNoNulo(s, i)$
        }
    }
}

\aux{dist}{x: \ent,
    y: \ent}{\ent}{
    abs(x-y)
}

\aux{indiceAnteriorNoNulo}{s: $señal$,
    i: \ent}{\ent}{
    \sum_{p=0}^{i-1}
    \IfThenElse{esElPrimerAnteriorNoNulo(s, i, p)}{p}{0}
}

\pred{esElPrimerAnteriorNoNulo}{s: $señal$,
    i: \ent,
    p: \ent}
{
    \paraTodo{j}{p}{i}
    \implicaLuego
    (s[j] = 0) \yLuego (s[p] \neq 0)
}

\aux{indiceSiguienteNoNulo}{s: $señal$,
    i: \ent}{\ent}
{
    \sum_{p=i+1}^{\longitud{s}-1}
    \IfThenElse{esElPrimerSiguienteNoNulo(s, i, p)}{p}{0}
}

\pred{esElPrimerSiguienteNoNulo}{s: $señal$,
    i: \ent,
    p: \ent}
{
    \paraTodo{j}{i}{p}
    \implicaLuego
    (s[j] = 0) \yLuego (s[p] \neq 0)
}

\aux{el2doIndiceNoNuloMasCercano}{s: $señal$,
    i: \ent}{\ent}{

    $elIndiceNoNuloMasCercano(setAt(s, elIndiceNoNuloMasCercano(s, i), 0))$
}


\pred{enDondeNoSeaCeroDebenCoincidir}{original: señal,
    reconstruida: señal}
{

    \paraTodo{i}{0}{\longitud{original}}
    \implicaLuego

        ($original[i]$ \neq 0) \yLuego 

        $(original[i] = reconstruida[i])$
}

% \end{document}