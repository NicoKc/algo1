\documentclass{article}
\usepackage{listings} 
\usepackage{multirow}
\input{../../../../recursos/latex/Algo1Macros.tex}

\lstset
{
    language=Caml,
    basicstyle=\footnotesize,
    numbers=left,
    stepnumber=1,
    showstringspaces=false,
    tabsize=1,
    breaklines=true,
    breakatwhitespace=false,
}

\begin{document}

\section*{Ejercicio 1}

\lstinputlisting{./code/max.cpp}

\begin{itemize}
    \item test1:
        \begin{itemize}
            \item entrada x = 0, y = 0
            \item Resultado esperado result = 0
        \end{itemize}
    \item test2:
        \begin{itemize}
            \item entrada x = 0, y = 1
            \item Resultado esperado result = 1
        \end{itemize}
\end{itemize}

\begin{enumerate}
    \item Diagrama de flujo

    \quad\includegraphics[scale=0.65]{./recursos/flow-chart.png}

    \item Líneas del programa que cubre cada test
    
    \begin{center}
        \begin{tabular}{ |c|c|c|c|c|c| } 
        \hline
        Test &  2   & 3     & 4     & 6     & 7     \\
        \hline
        test1 & Si  & Si    & No    & Si    & Si    \\
        test2 & Si  & Si    & Si    & No    & Si    \\
        \hline
        \end{tabular}
    \end{center}

    \item Decisiones

    \begin{center}
        \begin{tabular}{ |c|c|c| } 
        \hline
        Test  & branch true & branch false \\
        \hline
        test1 & No          & Si            \\
        test2 & Si          & No            \\
        \hline
        \end{tabular}
    \end{center}

    \item El test suit cubre 100\% de las líneas del programa y los branches

\end{enumerate}

\end{document}