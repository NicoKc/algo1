\documentclass{article}
\usepackage{listings}
\usepackage{xcolor}
\usepackage[shortlabels]{enumitem}
\input{../../../../recursos/latex/Algo1Macros.tex}

\definecolor{codegreen}{rgb}{0,0.6,0}
\definecolor{codegray}{rgb}{0.5,0.5,0.5}
\definecolor{codepurple}{rgb}{0.58,0,0.82}
\definecolor{backcolour}{rgb}{0.95,0.95,0.92}

\lstset
{
    language=Caml,
    backgroundcolor=\color{backcolour},
    commentstyle=\color{codegreen},
    keywordstyle=\color{magenta},
    numberstyle=\tiny\color{codegray},
    stringstyle=\color{codepurple},
    basicstyle=\ttfamily\footnotesize,
    breakatwhitespace=false,
    breaklines=true,
    captionpos=b,
    keepspaces=true,
    numbers=left,
    numbersep=5pt,
    showspaces=false,
    showstringspaces=false,
    showtabs=false,
    tabsize=2
}

\begin{document}
    \section*{Ejercicio 8}

    \lstinputlisting{./code/burbujeo.cpp}
    
    \begin{enumerate}[a)]
        \item Compara los ítems adyacentes e intercambia los que no están en orden.\\
        Cada pasada a lo largo de la listaubica el siguiente valor más grande en su lugar apropiado.\\
        En esencia, cada ítem “burbujea” hasta el lugar al que pertenece.
        \item $0 \leq i \leq \longitud{a} - 1 \yLuego \paraTodo{j}{\longitud{a}-i}{\longitud{a}} \implicaLuego a[j-1] \leq a[j] $
        \item $0 \leq j \leq \longitud{a} - 1 \yLuego \paraTodo{k}{0}{j} \implicaLuego a_0[k] > a_0[k+1] \implica a[k] = a_0[k+1] \wedge a[j+1] = a_0[j] $
        \item $ O(n^2) $
    \end{enumerate}

\end{document}